\documentclass[11pt]{article}

\usepackage{times,mathptm}
\usepackage{pifont}
\usepackage{exscale}
\usepackage{latexsym}
\usepackage{amsmath}
\usepackage{epsfig}

\textwidth 6.5in
\textheight 9in
\oddsidemargin -0.0in
\topmargin -0.0in

\parindent 0pt     % How much the first word of a paragraph is indented.
\parskip 0pt       % How much extra space to leave between paragraphs.

\begin{document}

\begin{center}             % If you only centering 1 line use \centerline{}
\begin{LARGE}
{\bf CS 312 Work Assignment 5}
\end{LARGE}
\vskip 0.25cm      % vertical skip (0.25 cm)

**Joseph Acosta**\\
\end{center}

\begin{enumerate}

\item
\begin{enumerate}
\item $4$\\
	  $5*6$\\
	  $8/2$\\
\item $S->    aS^1|bS^2$\\
	  $S^1->  aS^1|bS^1|aS^3$\\
	  $S^2->  aS^2|bS^2|bS^3$\\
	  $S^3->  \Sigma$\\
\item $S->	  bSaS^1|aS^2|aS^3|S^3$\\
	  $S^1->  aS^3$\\
	  $S^2->  bS^2aS^3|S^3$\\
	  $S^3->  \Sigma$\\
\item $S-> aSa|bSb|a|b|\Sigma$\\
\end{enumerate}

\item For each of the languages of (1), give an equivalent regular expression or explain
why the language is not regular.
\begin{enumerate}
\item Not Regular because we cannot keep track of the number of initial a's in order to double them after we have listed our number of b's. Since the a at the end is to the power of 2n and the a at the beginning is to the power of n, the a at the end must be listed twice as much. But it is impossible to keep track of the number of initial a's.\\
\item $(a+b)+(a^+(a+b)^*a^+)+(b^+(a+b)^*b^+)$\\
\item Not Regular because we cannot keep track of the number of b's without restricting the order of the letters.\\
\item Not Regular because we cannot keep track of the number of a's or b's at the first half to repeat at the second half.\\
\end{enumerate}

\item In the old days, scientific calculators and many programs took their input in what is known
as reverse polish notation. In this notation, the operator is written after the operands. For example,
the expression 3 4 + means (3 + 4) and the expression 3 4 + 7 * means (3 + 4) * 7. Consider the
following grammar for expressions in RPN:\\
\\
\begin{center}
\begin{enumerate}
\item This grammar is not ambiguous because it's impossible to have more than one derivation for the same string.\\
\item The advantage of RPN over in-fix is that RPN limits lookahead. You can perform the calculation once you hit the operator and there is no need to look ahead.\\
\vspace{12pt}
\item \\
\hspace{12pt}$\vdash int:int$\\
\underline{$\vdash S_1:int_1	\vdash S_2:int_2$}\\
$S_1S_2+:int_1+int_2$\\
\\
\vspace{12pt}
\underline{$\vdash S_1:int_1	\vdash S_2:int_2$}\\
$S_1S_2*:int_1*int_2$\\
\vspace{12pt}
\item  \\
\underline{$\vdash 3:3   \vdash 4:4$} \\
\underline{$\vdash 3\hspace{5pt} 4+:7$\hspace{40pt}$\vdash 2:2$}\\
$\hspace{30pt}\vdash 7\hspace{5pt}2*:14$\\

\end{enumerate}
\end{center}
\end{document}