\documentclass[11pt]{article}

\usepackage{times,mathptm}
\usepackage{pifont}
\usepackage{exscale}
\usepackage{latexsym}
\usepackage{amsmath}
\usepackage{epsfig}

\textwidth 6.5in
\textheight 9in
\oddsidemargin -0.0in
\topmargin -0.0in

\parindent 0pt     % How much the first word of a paragraph is indented.
\parskip 0pt       % How much extra space to leave between paragraphs.

\begin{document}

\begin{center}             % If you only centering 1 line use \centerline{}
\begin{LARGE}
{\bf CS 312 Work Assignment 2}
\end{LARGE}
\vskip 0.25cm      % vertical skip (0.25 cm)

**Joseph Acosta**\\
**Collaborated with Timothy Cohen**\\
\end{center}

\begin{enumerate}

\item Write regular expressions for the following languages over alphabet $/sigma = {a,b}$.
\begin{enumerate}
\item $(\lambda $x.$\lambda$y.y)22 5 \hspace{12pt}\\
(($\lambda$y.y) [22/x])5\\
($\lambda$y.y) 5\\
y[5/y]\\
5
\item $(\lambda $x.x)($\lambda$y.y)0 \hspace{12pt}\\
$(\lambda $x.x)y[0/y]\\
x[0/x]\\
0
\item $(\lambda$w.$\lambda$z.z+w) 97\hspace{12pt}\\
$\lambda$z.z+w[97/w]\\
$\lambda$z.z + 97\\
\item $(\lambda$a.$\lambda$b.a)$(\lambda$a.$\lambda$b.b)($\lambda$x.x) \hspace{12pt}\\
$(\lambda$a.$\lambda$b.a) $\lambda$b.b[$\lambda$x.x/a]\\
$(\lambda$a.$\lambda$b.a) $\lambda$b.b\\
$\lambda$b.a[$\lambda$b.b/a]\\
$\lambda$b.b.b\\

\item $(\lambda$x.x x x)$(\lambda$x.x x x) \hspace{12pt}\\
xxx[$\lambda$x.xxx/x]\\
$(\lambda$x.x x x)$(\lambda$x.x x x)$(\lambda$x.x x x)\\
$(\lambda$x.x x x) xxx[$(\lambda$x.x x x)/x]\\
$(\lambda$x.x x x)$(\lambda$x.x x x)$(\lambda$x.x x x)$(\lambda$x.x x x)\\
divergent\\

\item ($\lambda$a.($\lambda$b.b b)($\lambda$c.c c)) \hspace{12pt} \\
$\lambda$a. bb[$\lambda$c.cc/b]\\
($\lambda$a.($\lambda$c.cc)($\lambda$c.cc))\\
$\lambda$a.  cc[$\lambda$c.cc/c]\\
$\lambda$a.($\lambda$c.cc)($\lambda$c.cc)($\lambda$c.cc)\\
divergent

\end{enumerate}

\item Prove that the following $\lambda$ expressions are semantically equivalent:
\begin{enumerate}
\item ($\lambda$x.$\lambda$y.x + y) 44 and ($\lambda$b.44 + b)\hspace{12pt}\\

($\lambda$x.$\lambda$y.x + y) 44\\
Do $\beta$ reduction first\\
$\lambda$y.x + y[44/x]\\
$\lambda$y.44 + y\\
Do $\alpha$ reduction\\
$\lambda$b.44 + b\\
This is equal to ($\lambda$b.44 + b), therefore they are semantically equivalent.

\item $\lambda$x.$\lambda$y.y and $\lambda$a.$\lambda$b.b\hspace{12pt}\\

$\lambda$x.$\lambda$y.y\\
Do $\alpha$ reduction\\
$\lambda$a.$\lambda$y.y\\
Do $\alpha$ reduction\\
$\lambda$a.$\lambda$b.b\\
This is equal to $\lambda$a.$\lambda$b.b, therefore they are semantically equivalent.
\end{enumerate}


\item In lecture, we saw that it is possible to encode recursion in $\lambda$-calculus using fixed-point operators. We also studied one such operator, the Y-combinator.\\
After learning about the Y-combinator in lecture, a student in CS312 proposes the following "simpler" fixed-point operator:\\
\begin{center}
$\lambda$y.($\lambda$x.y (x x))\\
\end{center}
Recall that any fixed point operator must have the property that v=h(v) for any function h. Is the proposed construct a fixed-point operator or not? Prove your answer.\\

v = $\lambda$y.($\lambda$x.y (x x))\\
does v(h)=v?\\
v(h) = ($\lambda$y.($\lambda$x.y (x x)))h\\
$\lambda$x.y (x x)[h/y]\\
$\lambda$x.h (x x)\\
This is not the same as v, so it is not a fixed point operator.\\

\item In this question you will write three different programs that compute the factorial of 5 in L:\\

\begin{enumerate}
\item Write a program in L that computes the factorial of 5\\

fun fact with n = \\
if n$<=$0 \\
then 1 \\
else \\
n* (fact (n-1)) in\\ 
(fact 5)\\

\item Write the same program without using the function definition\\

let fact = lambda n. if n=0 then\\
1 \\
else \\
n* (fact(n-1)) in \\
(fact 5)\\

\item Write the same program without function definitions or let bindings (hint: the Y-combinator may be useful)\\

((lambda f.(lambda x.f(x x))lambda x.f(x x)) \\
(lambda f.lambda n.( if n=0 then 1 else n*(f(n-1))))) 5\\
\end{enumerate}
\item Write a function in L that, applied to a list, returns the length of this list. For example your function should return "2" for the list 33@5, 0 for the list Nil and 3 for the list "cs312"@"is"@"fun"\\

(Note: in this program you can replace "33@5" with whatever list you want)\\

fun length with list =\\ 
if isNil list then 0 else (length #list)+1\\
in \\
(length 33@5)\\

\item Write a function in L that reverses a list. Be careful! The reversal of\\

[1,[2,3]]\\
is [3, [2,1]]. not [[3,2],1].\\

Hint: You will probably fist need to write a list concatenation helper function.\\

(Note: in this program you can replace "1@2@3@4@5" with whatever list you want)\\

fun cat with l1, l2 =\\
if isNil l1 then l2 else\\
!l1@(cat #l1 l2)\\
in\\

fun rev with list =\\ 
if isNil list then Nil else\\
let h = !list in\\
let t = #list in\\
(cat (rev t) h)\\
in\\
(rev 1@2@3@4@5)\\
\end{enumerate}

\end{document}
