\documentclass[11pt]{article}

\usepackage{times,mathptm}
\usepackage{pifont}
\usepackage{exscale}
\usepackage{latexsym}
\usepackage{amsmath}
\usepackage{epsfig}

\textwidth 6.5in
\textheight 9in
\oddsidemargin -0.0in
\topmargin -0.0in

\parindent 0pt     % How much the first word of a paragraph is indented.
\parskip 0pt       % How much extra space to leave between paragraphs.

\begin{document}

\begin{center}             % If you only centering 1 line use \centerline{}
\begin{LARGE}
{\bf CS 312 Work Assignment 5}
\end{LARGE}
\vskip 0.25cm      % vertical skip (0.25 cm)

**Joseph Acosta**\\
\end{center}

\begin{enumerate}

\item
\begin{enumerate}
\item $4$\\
	  $5*6$\\
	  $8/2$\\
\item $\vdash S_1:c_1$\\

		\underline{$\vdash S_1:c_1$ \hspace{12} $\vdash S_2:c_2$}\\
		$\vdash S_1*S_2$ : $c_1*c_2$\\
		
		$\vdash S_2:$ $c_2$ is not zero\\
		\underline{$\vdash S_1:c_1$}\\
		$\vdash S_1/S_2 : c_1/c_2$\\
		
		\underline{$\vdash e_2[e_1/x]:e$}\\
		$\vdash$ let x = $e_1$ in $e_2$: e\\
		
\item \underline{$\Gamma\vdash S_1:non zero$ \hspace{12} $\Gamma\vdash S_2:non zero$}\\
		$\Gamma\vdash S_1*S_2:non zero$\\
		
		\underline{$\Gamma\vdash S_1:non zero$ \hspace{12} $\Gamma\vdash S_2:zero$}\\
		$\Gamma\vdash S_1*S_2:zero$\\
		
		\underline{$\Gamma\vdash S_1:zero$ \hspace{12} $\Gamma\vdash S_2:non zero$}\\
		$\Gamma\vdash S_1*S_2:zero$\\
		
		\underline{$\Gamma\vdash S_1:zero$ \hspace{12} $\Gamma\vdash S_2:zero$}\\
		$\Gamma\vdash S_1*S_2:zero$\\
		
		\underline{$\Gamma\vdash S_1:zero$ \hspace{12} $\Gamma\vdash S_2:non zero$}\\
		$\Gamma\vdash S_1/S_2:zero$\\
		
		\underline{$\Gamma\vdash S_1:non zero$ \hspace{12} $\Gamma\vdash S_2:non zero$}\\
		$\Gamma\vdash S_1/S_2:non zero$\\
		
		$\Gamma\vdash S_1: t_1$\\
		$t = t_1$\\
		\underline{$\Gamma[x \leftarrow t]\vdash S_2:t_2$ }\\
		$\Gamma\vdash let x=S_1 in S_2:T_2$\\
	
\item  \textbf{Base:}\\
		\\
			\underline{$non zero$ $i$}\\
			$\Gamma\vdash i:non zero$\\
			
			\underline{$zero$ $i$}\\
			$\Gamma\vdash i:zero$\\
			\\
			\textbf{Inductive Case 1:}\\
			\\
			\underline{$\vdash S_1:c_1$ \hspace{12} $\vdash S_2:c_2$}\\
			$\vdash S_1*S_2$ : $c_1*c_2$\\
			
			\underline{$\Gamma\vdash S_1:non zero$ \hspace{12} $\Gamma\vdash S_2:non zero$}\\
			$\Gamma\vdash S_1*S_2:non zero$\\
		
			\underline{$\Gamma\vdash S_1:non zero$ \hspace{12} $\Gamma\vdash S_2:zero$}\\
			$\Gamma\vdash S_1*S_2:zero$\\
		
			\underline{$\Gamma\vdash S_1:zero$ \hspace{12} $\Gamma\vdash S_2:non zero$}\\
			$\Gamma\vdash S_1*S_2:zero$\\
		
			\underline{$\Gamma\vdash S_1:zero$ \hspace{12} $\Gamma\vdash S_2:zero$}\\
			$\Gamma\vdash S_1*S_2:zero$\\
			
			Since we know $c_1$ is a non zero or a zero, and $c_2$ is a non zero or a zero we know $c_1*c_2$ is also non zero or zero.\\
			\\
			\textbf{Inductive Case 2:}\\
			\\
			\underline{$\vdash S_1:c_1$ \hspace{12} $\vdash S_2:c_2$}\\
			$\vdash S_1/S_2$ : $c_1/c_2$\\
			
			\underline{$\Gamma\vdash S_1:zero$ \hspace{12} $\Gamma\vdash S_2:non zero$}\\
		$\Gamma\vdash S_1/S_2:zero$\\
		
		\underline{$\Gamma\vdash S_1:non zero$ \hspace{12} $\Gamma\vdash S_2:non zero$}\\
		$\Gamma\vdash S_1/S_2:non zero$\\
		
		Since we know $c_1$ is a zero or a non zero, and we know $c_2$ is a non zero, $c_1/c_2$ must be a non zero.\\
		\\
		\textbf{Inductive Case 3:}\\
		\\
		\underline{$\vdash e_2[e_1/x]:e$}\\
		$\vdash$ let x = $e_1$ in $e_2$: e\\
		
		$\Gamma\vdash S_1: t_1$\\
		$t = t_1$\\
		\underline{$\Gamma[x \leftarrow t]\vdash S_2:t_2$ }\\
		$\Gamma\vdash let x=S_1 in S_2:T_2$\\
		
		Since we know $c_1$ is zero or non zero, we know $c_2$ must also be zero or non_zero.\\
	
		
\item \textbf{Base:}\\
		\\
			\underline{$non zero$ $i$}\\
			$\Gamma\vdash i:non zero$\\
			
			\underline{$zero$ $i$}\\
			$\Gamma\vdash i:zero$\\
			\\
			
			Clearly, if i types as zero or non zero, the corresponding operational
semantics rule applies\\
			\textbf{Inductive Case 1:}\\
			\\
			\underline{$\vdash S_1:c_1$ \hspace{12} $\vdash S_2:c_2$}\\
			$\vdash S_1*S_2$ : $c_1*c_2$\\
			
			\underline{$\Gamma\vdash S_1:non zero$ \hspace{12} $\Gamma\vdash S_2:non zero$}\\
			$\Gamma\vdash S_1*S_2:non zero$\\
		
			\underline{$\Gamma\vdash S_1:non zero$ \hspace{12} $\Gamma\vdash S_2:zero$}\\
			$\Gamma\vdash S_1*S_2:zero$\\
		
			\underline{$\Gamma\vdash S_1:zero$ \hspace{12} $\Gamma\vdash S_2:non zero$}\\
			$\Gamma\vdash S_1*S_2:zero$\\
		
			\underline{$\Gamma\vdash S_1:zero$ \hspace{12} $\Gamma\vdash S_2:zero$}\\
			$\Gamma\vdash S_1*S_2:zero$\\
			
			We know from the inductive hypothesis that the evaluation of
$S_1$ and $S_2$ will never get stuck. We also know from
preservation that the expressions $S_1$ and $S_2$ must evaluate to
zero or non zero, therefore the operational
semantics rule for multiplication will always apply since the hypotheses
only require that $c_1$ and $c_2$ are zero or non zero.\\
			\\
			\textbf{Inductive Case 2:}\\
			\\
			\underline{$\vdash S_1:c_1$ \hspace{12} $\vdash S_2:c_2$}\\
			$\vdash S_1/S_2$ : $c_1/c_2$\\
			
			\underline{$\Gamma\vdash S_1:zero$ \hspace{12} $\Gamma\vdash S_2:non zero$}\\
		$\Gamma\vdash S_1/S_2:zero$\\
		
		\underline{$\Gamma\vdash S_1:non zero$ \hspace{12} $\Gamma\vdash S_2:non zero$}\\
		$\Gamma\vdash S_1/S_2:non zero$\\
		
		We know from the inductive hypothesis that the evaluation of
$S_1$ and $S_2$ will never get stuck. We also know from
preservation that the expression $S_1$ must evauate to zero or non zero and $S_2$ must evaluate to non zero, therefore the operational
semantics rule for division will always apply since the hypotheses
only require that $c_1$ is a non zero or zero and $c_2$ is a non zero.\\
		\\
		\textbf{Inductive Case 3:}\\
		\\
		\underline{$\vdash e_2[e_1/x]:e$}\\
		$\vdash$ let x = $e_1$ in $e_2$: e\\
		
		$\Gamma\vdash S_1: t_1$\\
		$t = t_1$\\
		\underline{$\Gamma[x \leftarrow t]\vdash S_2:t_2$ }\\
		$\Gamma\vdash let x=S_1 in S_2:T_2$\\
		
		Here, we know from the inductive hypothesis that $S_1$ is of type $t_1$ and E[x \leftarrow t] $S_2$ : $t_2$ will not get stuck since they are well-typed. Therefore, this rule will also always apply.\\
		
			
			

\end{enumerate}

\item 
\begin{enumerate}
\item These are large step Operational Semantics.\\
\item TU 'd' \rightarrow 'D'\\
	TU 'a' \rightarrow 'A' \\
		TL 'B' \rightarrow 'b'\\
		\item 
		\\
\item 1. A char is defined as a char.\\
		2. AL returns a lowercase char if it's the correct case.\\
		3. AU returns an uppercase char if it's the correct case.\\
		4. TU converts a char to uppercase and returns it.\\
		5. TL converts a char to lowercase and returns it.\\
		\\
\item A runtime error will occur if AL is called on an uppercase char. A runtime error will also occur if AU is called on a lowercase char.\\

\item A type system to prevent runtime errors for this language would consist of the types lowercase and uppercase. \\
The concretization of the type lowercase is: $\gamma(lowercase)$ = 'a'-'b'.\\
 The concretization of the type uppercase is: $\gamma(uppercase)$ = 'A'-'B'.\\
 
\item \underline{lowercase c}\\
		$\Gamma\vdash c:c$\\
		
		\underline{lowercase c}\\
		$\Gamma\vdash c:c$\\
		
		\underline{$\Gamma\vdash P:lowercase $}\\
		$\Gamma\vdash AL\hspace*{3} P: lowercase$\\
		
		\underline{$\Gamma\vdash P:uppercase $}\\
		$\Gamma\vdash AU\hspace*{3} P: uppercase$\\
		
		\underline{$\Gamma\vdash P:lowercase $}\\
		$\Gamma\vdash TL\hspace*{3} P: lowercase$\\
		
		\underline{$\Gamma\vdash P:uppercase $}\\
		$\Gamma\vdash TL\hspace*{3} P: lowercase$\\
		
		\underline{$\Gamma\vdash P:uppercase $}\\
		$\Gamma\vdash TU\hspace*{3} P: uppercase$\\
		
		\underline{$\Gamma\vdash P:lowercase $}\\
		$\Gamma\vdash TU\hspace*{3} P: uppercase$\\
		
 

\end{enumerate}

\end{enumerate}
\end{document}